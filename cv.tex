%!TEX TS-program = xelatex
%!TEX TS-options = --shell-escape

\documentclass[]{article}
\usepackage[french]{babel}
\usepackage{fancyhdr}
\usepackage{titlesec}
\usepackage{titling}
\usepackage{color}
\usepackage{fontspec,xltxtra,xunicode}
\usepackage{tikz}
\usepackage[left=6.1cm,top=2cm,right=1.5cm,bottom=2.5cm,nohead,nofoot]{geometry}
\usepackage[absolute,overlay]{textpos}
%\usepackage[colorgrid,texcoord]{eso-pic}
\usepackage{parskip}
\usepackage{hyperref}

\setlength{\TPHorizModule}{1cm}
\setlength{\TPVertModule}{1cm}

\definecolor{darkgray}{RGB}{51,51,51}
\definecolor{gray}{RGB}{77,77,77}
\definecolor{lightgray}{RGB}{153,153,153}
\definecolor{white}{RGB}{255,255,255}
\definecolor{green}{RGB}{181,226,70}

\newfontfamily\headingfont[]{Helvetica Neue Condensed Bold}
\titleformat*{\section}{\LARGE\headingfont\color{gray}}
\titleformat*{\subsection}{\Large\headingfont\color{gray}}
\titleformat*{\subsubsection}{\large\headingfont\color{gray}}

\defaultfontfeatures{Mapping=tex-text}
\setromanfont[Mapping=tex-text, Color=5f5f5f]{Helvetica Neue Light}
\setsansfont[Scale=MatchLowercase,Mapping=tex-text]{Gill Sans}
\setmonofont[Scale=MatchLowercase]{Andale Mono}

\renewcommand{\bibliography}{
  \immediate\write18{/usr/local/bin/env coffee biblio.coffee}
  \input{biblio.tex}
}

\thispagestyle{empty}
\pagestyle{empty}

\newcommand{\sideheader}[1]{
  \vspace{\baselineskip}{\Large\headingfont\color{gray} #1}
}
\makeatletter
\renewcommand{\section}{
  \@startsection{section}{1}{0mm}
    {\parskip}
    {\lineskip}
    {\LARGE\headingfont\color{gray}}
}
\makeatother
\makeatletter
\renewcommand{\subsection}{
  \@startsection{subsection}{1}{0mm}
    {0mm}
    {\parsep}
    {\headingfont\large\color{gray}}
}
\makeatother

\newcommand{\btitle}[1]{
  \par
  {\fontsize{10pt}{18pt}\selectfont #1}\\
}
\newcommand{\bauthors}[1]{
  {\fontsize{9pt}{18pt}\addfontfeatures{Color=lightgray}\selectfont #1}\\
}
\newcommand{\bmore}[1]{
  \emph{\fontsize{8pt}{18pt}\addfontfeatures{Color=lightgray}\selectfont #1}
}

\newcommand{\green}[1]{{\addfontfeatures{Color=green}#1}}
\newcommand{\orange}[1]{{\addfontfeatures{Color=fda333}#1}}
\newcommand{\purple}[1]{{\addfontfeatures{Color=d3a4f9}#1}}
\newcommand{\red}[1]{{\addfontfeatures{Color=fb4485}#1}}
\newcommand{\blue}[1]{{\addfontfeatures{Color=6ce0f1}#1}}
\newcommand{\gray}[1]{{\addfontfeatures{Color=lightgray}#1}}

\newcommand{\row}[3]{#1&\parbox[t]{11.8cm}{{#2}\\#3\vspace{\parsep}}\\\noalign{\smallskip}}
\newcommand{\bold}[1]{{\headingfont\addfontfeatures{Color=gray}\normalsize{#1}}}
\newcommand{\light}[1]{{\fontspec{Helvetica Neue Light}\color{lightgray}\footnotesize{#1}}}

\setlength{\tabcolsep}{0pt}

\begin{document}
  \begin{tikzpicture}[remember picture,overlay]
    \node [rectangle, fill=gray, anchor=north, minimum width=\paperwidth, minimum height=4cm] (box) at (current page.north){};
    \node [anchor=center] (name) at (box) {
      \fontsize{40pt}{72pt}\selectfont
      \fontspec{Helvetica Neue UltraLight}
      \color{white}adrien\fontspec{Helvetica Neue}friggeri
    };
    \node [anchor=north] at (name.south) {
      \fontsize{14pt}{24pt}\selectfont
      \fontspec{Helvetica Neue UltraLight}
      \color{white} social network analyst
    };
  \end{tikzpicture}
  \vspace{2.5cm}
  
  \begin{textblock}{3.6}(1.5, 4.34)
    \begin{flushright}
      \sideheader{about}\\
      31 rue Smith\\
      69002 Lyon\\
      France\\
      +33 6.73.51.32.75\\
      ~ \\
      \shorthandoff{:}
      \href{mailto:adrien@friggeri.net}{adrien@friggeri.net}\\
      \href{http://friggeri.net}{http://friggeri.net}\\
      \href{http://facebook.com/adrien}{fb://adrien}\\
      \shorthandon{:}
      \sideheader{languages}\\
      bilingual french/english\\
      spanish \& italian notions\\

      \sideheader{programming}\\
      $\heartsuit$ JavaScript\\
      (ES5, node.js)\\
      Python, C, OCaml\\
      CSS3 \& HTML5\\ 
    \end{flushright}
  \end{textblock}
  \vspace{-2\parskip}
  \section*{\gray{int}erests}
  complex networks, social networks, community detection, community structure,
  overlapping communities, information diffusion, viral marketing, social
  inference, recommendation, data mining
  
  \section*{\blue{edu}cation}
  \begin{tabular*}{\textwidth}{@{\extracolsep{\fill}}ll}
    \row{since 2009}
      {\bold{Ph.D.} candidate in Computer Science \hfill \light{DNET/INRIA, LIP/ÉNS de Lyon}}
      {\emph{A Quantified Theory of Social Cohesion.}}
    \row{2007–2008}
      {\bold{M.Sc. magna cum laude} \hfill \light{IXXI, École Normale Supérieure de Lyon}}
      {Majoring in Computer Science\\
      Specialization in Complex Systems}
    \row{2006–2007}
      {\bold{B.Sc. magna cum laude} \hfill \light{École Normale Supérieure de Lyon}}
      {Majoring in Computer Science}
    \row{2003–2006}
      {\bold{Classes Préparatoires aux Grandes Écoles} \hfill \light{Lycée Fénelon, Lycée Louis le Grand, Paris}}
      {Preparation for national competitive entrance exams to leading French ``grandes écoles'', specializing in mathematics and physics.}
    \row{2003}
      {\bold{French Baccalauréat S. with honors} \hfill \light{Lycée Louis le Grand, Paris}}
      {Specialization in mathematics and physics}
  \end{tabular*}
  
  \section*{\red{exp}erience}

  \begin{tabular*}{\textwidth}{@{\extracolsep{\fill}}ll}
    \row{02–07/2009}{\bold{LIP6/CNRS, Paris}}{Research Internship. \emph{Visualization of complex networks.}}
    \row{06–08/2008}{\bold{ISCPIF/CNRS, Paris}}{Research Internship. \emph{Diffusion in the Blogosphere. Happy Flu.}}
    \row{06–08/2007}{\bold{LIP6/CNRS, Paris}}{Research Internship. \emph{Kernels in real world networks.}}
    \row{07–08/2005}{\bold{\href{http://www.kelkoo.com}{Kelkoo.com}}}{Summer job. \emph{Creation of a keyword generator for Google Adwords.}}
    \row{07–08/2004}{\bold{\href{http://www.monsieurprix.com}{MonsieurPrix.com}}}{Summer job. \emph{Development of an e-commerce product indexation spider.}}
  \end{tabular*}
  
  \section*{\green{onl}ine}
  \bold{2012 Who did I forget ?} \hfill
                                 \light{\href{http://whodidiforget.com}{whodidiforget.com}}\\
  Guest list recommandation for Facebook events based on friends already 
  attending the event.

  
  \bold{2011 Fellows} \hfill
                      \light{\href{http://fellows-exp.com}{fellows-exp.com}}\\
  Automatic community detection among Facebook Friends in order to validate
  the \emph{cohesion} measure, creation of friend lists.
  
  
  \bold{2008 Happy Flu}\hfill
                            \light{\href{http://happyflu.com}{happyflu.com}}\\
  Experiment aimed to measure viral spreading of content across the
  blogosphere.
  
  \section*{\orange{pub}lications}
  \bibliography
  
  \section*{\purple{pre}ss}
  
  \btitle{\href{http://www.allfacebook.com/facebook-event-app-2012-01}{Who Did I Forget? Facebook Application Has The Answer}}
  \bauthors{All Facebook}
  \bmore{January 31st, 2012}
  
  \btitle{\href{http://wisemetrics.com/blog/2012/01/power-of-the-graph/}{Power of the graph : auto-fill your event guest list by quoting only one person}}
  \bauthors{Wise Metrics}
  \bmore{January 31st, 2012}
  
  \btitle{\href{http://fr.locita.com/reseaux-sociaux/facebook/une-experience-sociale-sur-facebook-des-triangles-pour-mesurer-la-cohesion-sociale/}{Une expérience sociale sur Facebook : des triangles pour mesurer la cohésion sociale}}
  \bauthors{Locita}
  \bmore{January 31st, 2012}
  
  \btitle{\href{http://www.minutebuzz.com/2011/03/22/fellows-fait-le-tri-parmi-vos-contacts-facebook/}{Fellows fait le tri parmi vos contacts Facebook}}
  \bauthors{MinuteBuzz}
  \bmore{March 22nd, 2011}
  
  \btitle{\href{http://www.mediassociaux.fr/2011/03/07/a-quoi-sert-votre-graphe-social/}{A quoi sert votre graphe social ?}}
  \bauthors{MediaSociaux.fr}
  \bmore{March 7th, 2011}
  
  \btitle{\href{http://www.demainlaveille.fr/2011/03/02/fellows-met-de-lordre-dans-vos-contacts-facebook/}{Fellows met de l’ordre dans vos contacts Facebook}}
  \bauthors{Demain la veille}
  \bmore{March 2nd, 2011}
  
  \btitle{\href{http://compagnon-parfait.fr/un-outil-pour-trier-ses-amis-facebook-art-3042-1.html}{Un outil pour trier ses amis Facebook}}
  \bauthors{Compagnon Parfait}
  \bmore{March 1st, 2011}
  
  \btitle{\href{http://www.itrmanager.com/articles/115762/inria-lance-experimentation-reseau-social-facebook.html}{L’INRIA lance une expérimentation sur le réseau social Facebook}}
  \bauthors{ITR Manager}
  \bmore{February 28th, 2011}
  
  \btitle{\href{http://www.atelier.net/trends/articles/reseaux-ligne-communautes-sidentifient-automatiquement}{Sur les réseaux en ligne, les communautés s'identifient automatiquement}}
  \bauthors{L'Atelier}
  \bmore{February 23rd, 2011}
  
  \btitle{\href{http://socialiving.wordpress.com/2011/02/18/fellows-a-social-experiment/}{Fellows, a Social Experiment}}
  \bauthors{Social Living}
  \bmore{February 18th, 2011}
  
  \btitle{\href{http://www.inria.fr/centre/grenoble/actualites/experimentation-fellows-sur-facebook}{L'équipe DNET lance une expérimentation sur le réseau social Facebook}}
  \bauthors{INRIA}
  \bmore{February 16th, 2011}

  \btitle{\href{http://www.ens-lyon.eu/1297847447289/0/fiche___article/&RH=ENS-LYON-FR-RECH-FAI}{Fellows : une expérience sociale}}
  \bauthors{ENS Lyon}
  \bmore{February 16th, 2011}

\end{document}
