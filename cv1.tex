
%%%%%%%%%%%%%%%%%%%%%%%%%%%%%%%%%%%%%%%%%
% 
% K-CV -- Klenske Curriculum Vitae
% 
% Simplified Friggeri CV 
% by Edgar Klenske (ed.klenske@gmx.de)
% https://github.com/eklenske/CV
%
% Forked from:
% Jelmer Tiente 
% https://github.com/JelmerT/CV
%
% Based on original work by:
% Adrien Friggeri (adrien@friggeri.net)
% https://github.com/afriggeri/CV
%
% License:
% CC BY-NC-SA 3.0 (http://creativecommons.org/licenses/by-nc-sa/3.0/)
%
%%%%%%%%%%%%%%%%%%%%%%%%%%%%%%%%%%%%%%%%%

\documentclass[]{k-cv} % Add 'print' as an option into the square
                       % bracket to remove colors from this template
                       % for printing
\usepackage[T1]{fontenc}
\newcommand\tab[1][1cm]{\hspace*{#1}}


\begin{document}
\header{Oliver James }{Hall}{PhD student in asteroseismology} % Your name and current job

%-------------------------------------------------------------------------------
%	SIDEBAR SECTION
%-------------------------------------------------------------------------------

\begin{aside} % In the aside, each new line forces a line break
\section{programming}
Python (advanced)
Unix, LaTeX, Git (intermediate)
R, SQL/ADQL (basic)
\section{skills}
Stan
emcee
Bayesian statistics
Hierarchical models
Software development \& publication (Python)
\section{languages}
English, Dutch (bilingual)
\section{contact}
School of Physics \& Astronomy
University of Birmingham
B15 2TT
Birmingham
United Kingdom
~
\href{mailto:ojh251@student.bham.ac.uk}{ojh251@bham.ac.uk}
\href{http://www.ojhall94.github.io}{ojhall94.github.io}
\href{http://www.github.com/ojhall94}{GitHub/ojhall94}
\href{http://www.twitter.com/asteronomer}{@asteronomer}
\href{http://www.orcid.com/0000-0002-0468-4775}{ORCID/}
\href{http://www.orcid.com/0000-0002-0468-4775}{0000-0002-0468-4775}
\end{aside}

%-------------------------------------------------------------------------------
%	EDUCATION SECTION
%----------------------------------------------------------------------------------------
\section{research interests}
\bodyfont With the recent succes of the \textit{Kepler} and K2 missions, and the ongoing release of data from \textit{Gaia} and TESS, we are in posession of a vast amount of astronomical data. I am intersted in leveraging these large data sets to make inferences of stellar physics, \& analysis systematics. I do this through a Bayesian use of populations of asteroseismic data, in combination with other sources. I have used hierarchical models to study systematics and constrain the Red Clump standard candle to unprecedented precision. My current work focuses on studying the relation between mass, rotation and age of solar-like stars in \textit{Kepler} and K2 fields.

\section{education}

\begin{entrylist}
\entry
{2016 \to 2020}
{PhD {\normalfont in Physics \& Astronomy}}
{University of Birmingham, UK}
{Supervisor: Dr. Guy R. Davies\\
\emph{"Asteroseismology with \textit{Kepler}, K2 and TESS"} \vspace{0.2cm}}
%------------------------------------------------
%------------------------------------------------
\entry
{2012 \to 2016}
{M.Sci. {\normalfont Physics \& Astrophysics}}
{University of Birmingham, UK}
{Disseration supervisor: Prof. William J. Chaplin\\
\emph{"Detecting Signatures of Stellar Activity Cycles in Solar-Type Stars Using Asteroseismic Analysis of P-Mode Amplitude Shifts"} \\ 
1\textsuperscript{st} Class w. Honours\vspace{0.2cm}}
%------------------------------------------------


%------------------------------------------------
\entry
{2006 \to 2012}
{Gymnasium}
{Gemeentelijk Gymnasium Hilversum, Netherlands}
{9/10 in Maths, Physics and Chemistry\\
8.5/10 average across eleven subjects}
%------------------------------------------------
\end{entrylist}


\section{teaching and research}


\begin{entrylist}
%------------------------------------------------
\entry
{2017 \to now}
{2\textsuperscript{nd} Year Laboratory Projects Demonstrator}
{University of Birmingham, UK}
{Taught projects varying from spectroscopy to the building of a theremin. Helped students build apparatus, understand their results, and was responsible for marking their work and providing constructive feedback.}
%------------------------------------------------
\entry
{2016 \to now}
{3\textsuperscript{rd} Year Observatory Laboratory Supervisor}
{University of Birmingham, UK}
{Helped supervise students in their research using the University of Birmingham Observatory. Helped the students understand their results, as well as aiding them in the use of IRAF, Unix, LaTeX and Python.}
%------------------------------------------------
\entry
{2015}
{Summer Undergraduate Reserach Experience (SURE)}
{University of Leicester, UK}
{Was selected to perform a six-week project using Python to program a
Universal-Robots UR5 robotic arm system to perform careful experimantal testing
on a prototype focal plane for the Cherenkov Telescope Array under Dr. Jon
Lapington. Disseminated results through a report and group presentation.}
%------------------------------------------------
\entry
{2015}
{Ogden Trust Teach Physics Intern}
{Bishop Challoner Catholic College, Birmingham, UK}
{Was selected as one of the Ogden Trust's Teach Physics interns. I helped teach
pupils throughout lessons and prepared, taught a lesson \& careers workshop of my own
design.}
%------------------------------------------------
\end{entrylist}

\clearpage
\smallheader{oliver james  }{hall}
%-------------------------------------------------------------------------------
%	AWARDS SECTION
%-------------------------------------------------------------------------------
\section{grants \& awards}

\begin{entrylist}
%------------------------------------------------
\entrythree
{2018}
{IOP Research Student Conference Fund - \pounds 300 \emph{(declined)}}
{Institute of Physics, UK}

%------------------------------------------------
\entrythree
{2016}
{Royal Society Partnership Grant - \pounds 3000}
{The Royal Society, UK}
%------------------------------------------------
\entrythree
{2015}
{Teach Physics Oustanding Intern 2015 - shortlisted}
{The Ogden Trust}
%------------------------------------------------

\end{entrylist}



%-------------------------------------------------------------------------------
%	COMMUNICATION SKILLS SECTION
%-------------------------------------------------------------------------------

\section{presentations}

\begin{entrylist}
%------------------------------------------------
\entry
{2018 Dec.}
{Birmingham-Warwick Science Meet-Up}
{University of Warwick, UK}
{\emph{"Testing asteroseismology with \textit{Gaia} DR2: Hierarchical Models \& the Red Clump"}}
%------------------------------------------------
\entry
{2018 Jul.}
{TASC4/KASC11}
{Aarhus University, Denmark}
{\emph{"Testing asteroseismology with \textit{Gaia} DR2: Luminosity of the Red Clump"}}
%------------------------------------------------
\entry
{2017 Jul.}
{TASC3/KASC10 (poster presentation)}
{University of Birmingham, UK}
{\emph{"Mixture Models applied to \emph{Kepler} backgrounds \& development for TESS}}
%------------------------------------------------
\entry
{2017 Apr.}
{T'DA 2}
{Aarhus University, Denmark}
{\emph{"Estimating TESS backgrounds with mixture models -- Update"}}
%------------------------------------------------
\entry
{2016 Nov.}
{T'DA 1}
{University of Birmingham, UK}
{\emph{"Estimating TESS backgrounds with mixture models"}}
%------------------------------------------------
\end{entrylist}

\section{conferences \& research visits}

\begin{entrylist}
%------------------------------------------------
\entrythree
{2019 Jan.}
{T'DA 8}
{Aarhus University, Denmark}
%------------------------------------------------
\entrythree
{2018 Oct.}
{T'DA 5}
{Ohio State University, OH, USA}
%------------------------------------------------
\entrythree
{2018 Oct.}
{3 week research visit to the KeplerGO office}
{NASA Ames Research Centre, CA, USA}
%------------------------------------------------
\entrythree
{2018 Jul.}
{T'DA 4}
{Aarhus University, Denmark}
%------------------------------------------------
\entrythree
{2018 Jul.}
{TASC4/KASC11}
{Aarhus University, Denmark}
%----------------------------------------------
\entrythree
{2018 Jan.}
{1 week research visit to SAC}
{Aarhus University, Denmark}
%------------------------------------------------
\entrythree
{2017 Dec.}
{T'DA 3}
{KU Leuven, Belgium}
%------------------------------------------------
\entrythree
{2017 Jul.}
{TASC3/KASC10}
{University of Birmingham, UK}
%------------------------------------------------
\entrythree
{2017 Apr.}
{T'DA 2}
{Aarhus University, Denmark}
%------------------------------------------------
\entrythree
{2016 Nov.}
{Asteroseismology of stellar activity cycles}
{Observatoire de la C\^{o}te d'Azur, France}
%------------------------------------------------
\entrythree
{2016 Nov.}
{T'DA 1}
{University of Birmingham, UK}
%------------------------------------------------
\end{entrylist}

\section{outreach \& engagement}

\begin{entrylist}
%------------------------------------------------
\entry
{2019 \to now}
{\emph{Author}, Astrobites Collaboration}
{}
{Write and edit monthly summaries of astronomy papers for an undergraduate level for the website Astrobites.}
%------------------------------------------------
\entry
{2019 Jan.}
{\emph{Featured Astonomer}, Astrotweeps}
{}
{Hosted the @astrotweeps Twitter account for a week, providing public-level explanations of asteroseismology and space-based photometry.}
%------------------------------------------------
\entry
{2018 \to 2019}
{\emph{LOC \& SOC}, 9\textsuperscript{th} BEAR PGR Conference}
{University of Birmingham, UK}
{Organised local annual high performance computing conference.}
%------------------------------------------------
\entry
{2018 \to now}
{\emph{Demonstrator}, Applicant Visit Day}
{University of Birmingham, UK}
{Developed and taught laboratory sessions for undergraduate applicants.}
%------------------------------------------------
\entry
{2016 \to 2017}
{\emph{Partnered Researcher}, Royal Society Partnership Grant}
{}
{Developed and taught a series of lessons and lab activities engaging Year 9 pupils with exoplanet characterisation and asteroseismology.}
\end{entrylist}

\clearpage
\smallheader{oliver james  }{hall}

\section{community services}
\begin{entrylist}
%------------------------------------------------
\entrythree
{2018 \to now}
{Member of the \texttt{lightkurve} collaboration}
{NASA Ames Research Centre, CA, USA}
%------------------------------------------------
\entrythree
{2017}
{LOC member for TASC3/KASC11}
{University of Birmingham, UK}
%------------------------------------------------
\entrythree
{2016 \to now}
{Member of the \emph{TESS Data for Asteroseismology} (T'DA) collaboration }
{}
%------------------------------------------------
\entrythree
{continuous}
{Publish eductional blogs and tutorials online}
{}
%------------------------------------------------
\end{entrylist}
\section{publications}
\bibentry{\textbf{Hall, O. J.}, Davies, G. R., Elsworth, Y. et al.}
{Testing asteroseismology with \textit{Gaia} DR2: Hierarchical models of the Red Clump}
{Submitted to MNRAS}
{}


\bibentry{{Bugnet}, L., {Garc{\'\i}a}, R. A., {Mathur}, S.,
         {Davies}, G. R., \textbf{{Hall}, O. J.}, {Lund}, M. N., {Rendle}, B. M.}
{FliPer$_{Class}$: In search of solar-like pulsators among TESS targets}
{arXiv e-prints, 2019}
{\texttt{arXiv:1902.09854}}

\bibentry{{Huber}, D, {Chaplin}, W. J., {Chontos}, A ... \textbf{{Hall}, O. J.} ... et al. [1 citation]}
{{A Hot Saturn Orbiting An Oscillating Late Subgiant Discovered by TESS}}
{arXiv e-prints, 2019}
{\texttt{arXiv:1901.01643}}

\bibentry{{Lightkurve Collaboration}, {Cardoso}, J. V. d. M., {Hedges}, C. ...       					\textbf{{Hall}, O. J.} ... et al. [2 citations]}
{Lightkurve: Kepler and TESS time series analysis in Python}
{Astrophysics Source Code Library, 2018}
{\texttt{ascl:1812.013}}

\bibentry{{Bugnet}, L., {Garc{\'\i}a}, R. A., {Davies}, G. R. ...
         \textbf{{Hall}, O. J.} ... et al. [7 citations]}
{FliPer: A global measure of power density to estimate surface gravities of main-sequence solar-like stars and red giants}
{Astronomy \& Astrophysics, 2018}
{\texttt{doi:0.1051/0004-6361/201833106, arXiv:1809.05105}}

\bibentry{{Khan}, S., \textbf{{Hall}, O. J.}, {Miglio}, A. et al. [5 citations]}
{The Red-giant Branch Bump Revisited: Constraints on Envelope Overshooting in a Wide Range of Masses and Metallicities}
{The Astrophysical Journal, 2018}
{\texttt{doi:10.3847/1538-4357/aabf90, arXiv:1804.06669}}

\bibentry{{Davies}, G. R., {Lund}, M. N. and {Miglio}, A. ... \textbf{Hall, O. J.} ... et al. [20 citations]}
{Using red clump stars to correct the \emph{Gaia} DR1 parallaxes}
{Astronomy \& Astrophysics, 2017}
{\texttt{doi:10.1051/0004-6361/201630066, arXiv:1701.02506}}

\end{document}
