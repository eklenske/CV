%%%%%%%%%%%%%%%%%%%%%%%%%%%%%%%%%%%%%%%%%
% 
% K-CV -- Klenske Curriculum Vitae
% 
% Simplified Friggeri CV 
% by Edgar Klenske (ed.klenske@gmx.de)
% https://github.com/eklenske/CV
%
% Forked from:
% Jelmer Tiente 
% https://github.com/JelmerT/CV
%
% Based on original work by:
% Adrien Friggeri (adrien@friggeri.net)
% https://github.com/afriggeri/CV
%
% License:
% CC BY-NC-SA 3.0 (http://creativecommons.org/licenses/by-nc-sa/3.0/)
%
%%%%%%%%%%%%%%%%%%%%%%%%%%%%%%%%%%%%%%%%%

\documentclass[letterpaper]{k-cv} % Add 'print' as an option into the square
                       % bracket to remove colors from this template
                       % for printing
\usepackage[T1]{fontenc}
\newcommand\tab[1][1cm]{\hspace*{#1}}
%0173b2, de8f05, 029e73, d55e00, cc78bc
\definecolor{c1}{HTML}{0173b2}
\pagenumbering{arabic}

\begin{document}
\header{Oliver James }{Hall}{ESA Research Fellow - Asteroseismology \& Statistics} % Your name and current job

%-------------------------------------------------------------------------------
%	SIDEBAR SECTION
%-------------------------------------------------------------------------------

\begin{aside} % In the aside, each new line forces a line break
\section{\color{c1}programming}
\bodyfont Python, Git
Unix, LaTeX,
SQL
\section{\color{c1}skills}
Stan, PyMC3, emcee 
Bayesian statistics
Hierarchical models
Asteroseismology
Jupyter Notebooks
Science Communication \& Writing
Software development \& publication
\section{\color{c1}languages}
English, Dutch (bilingual)
\section{\color{c1}contact}
European Space Research \& Technology Centre
Keplerlaan 1
Postbus 299
2200 AG Noordwijk
The Netherlands
~
\href{mailto:oliver.hall@esa.int}{oliver.hall@esa.int}
\href{www.asteronomer.com}{asteronomer.com}
\href{http://www.github.com/ojhall94}{GitHub/ojhall94}
\href{http://www.twitter.com/asteronomer}{@asteronomer}
\href{http://www.orcid.com/0000-0002-0468-4775}{ORCID/}
\href{http://www.orcid.com/0000-0002-0468-4775}{0000-0002-0468-4775}
Tel: (+31)(0)614227748
\end{aside}

\section{\color{c1}positions}
\begin{entrylist}
	\entry
	{2020 \to now}
	{ESA Research Fellow}
	{European Space Research \& Technology Centre, Netherlands}
	{$+$ Work on Bayesian ensemble analysis problems in asteroseismology and stellar astronomy\\
		$+$ Develop open-source software to elevate science of current and future ESA missions}
	
	\entry
	{2020}
	{\textbf{Freelance Developer}}
	{NumFOCUS, TX, USA}
	{$+$ Developed \href{https://github.com/spacetelescope/notebooks/blob/master/notebooks/MAST/Kepler/README.md}{training materials for \emph{Kepler} and K2 users} for STScI\\
		$+$ Worked closely with a global team of collaborators to both write training materials and develop \href{https://github.com/lightkurve/lightkurve}{Lightkurve} code}
	%------------------------------------------------
\end{entrylist}

%0173b2, de8f05, 029e73, d55e00, cc78bc
\definecolor{c1}{HTML}{de8f05}
\section{\color{c1}education}

\begin{entrylist}
	
	\entry
	{2016 \to 2020}
	{PhD {\bodyfont in Physics \& Astronomy}}
	{University of Birmingham, UK}
	{$+$ Supervisor: Dr. Guy R. Davies\\
		$+$ \textit{Thesis}: "Ensemble Asteroseismology and Hierarchical Bayesian Models: New Inferences of Astrophysics with Oscillating Stars"} 
	%------------------------------------------------
	%------------------------------------------------
	\entry
	{2012 \to 2016}
	{M.Sci. {\bodyfont in Physics \& Astrophysics}}
	{University of Birmingham, UK}
	{$+$ 1\textsuperscript{st} Class w. Honours\\ 
		$+$ Dissertation supervisor: Prof. William J. Chaplin\\
		$+$ Thesis topic: Detecting signatures of stellar activity cycles using asteroseismic frequency shifts}
	%------------------------------------------------
	%------------------------------------------------
	\entry
	{2006 \to 2012}
	{Gymnasium}
	{Gemeentelijk Gymnasium Hilversum, Netherlands}
	{$+$ 8.5/10 average across eleven subjects}
	%------------------------------------------------
\end{entrylist}

%0173b2, de8f05, 029e73, d55e00, cc78bc
\definecolor{c1}{HTML}{029e73}
\section{\color{c1}selected presentations}

\begin{entrylist}
%------------------------------------------------
\entry
{2021 Mar.}
{\textbf{SCI-S Science \textcolor{c1}{Seminar}}}
{Virtual - ESA}
{"Hierarchical models and asteroseismic rotation"}
%------------------------------------------------
\entry
{2021 Mar.}
{\textbf{SAC \textcolor{c1}{Seminar}}}
{Virtual - Aarhus University, Denmark}
{"Hierarchical models and asteroseismic rotation"}
%------------------------------------------------
\entry
{2020 Feb.}
{\textbf{CSH Symposium}}
{Centre for Space and Habitability, Switzerland}
{\textbf{\textcolor{c1}{Invited talk:}} "Asteroseismology \& Rotational Evolution: Bayesian Inference in Stellar Astrophysics"}
%------------------------------------------------
\entry
{2020 Jan.}
{\textbf{ESA Research Fellow Jamboree}}
{ESA ESTEC, The Netherlands}
{"Asteroseismic Follow-Up of \textit{CHEOPS} Target Hosts"}
%------------------------------------------------
\entry
{2019 Nov.}
{Departmental \textbf{\textcolor{c1}{Seminar}}}
{University of Exeter, UK}
{\href{https://speakerdeck.com/ojhall94/seminar-asteroseismology-and-applied-statistics}{"Asteroseismology \& Applied Statistics"}}
%------------------------------------------------
\entry
{2019 Jul.}
{TASC5/KASC12}
{MIT, MA, USA}
{\textbf{\textcolor{c1}{Invited talk:}} \href{https://speakerdeck.com/ojhall94/accessible-asteroseismology-with-lightkurve}{"Accessible Asteroseismology with Lightkurve"\\ \small{Poster: "Improving gyrochronology of field stars with asteroseismic age and rotation"}}}
%------------------------------------------------
\entry
{2018 Dec.}
{Birmingham-Warwick Science 
	Meet-Up}
{University of Warwick, UK}
{\href{https://speakerdeck.com/ojhall94/testing-asteroseismology-with-gaia-dr2-hierarchical-models-and-the-red-clump}{"Testing asteroseismology with \textit{Gaia} DR2: Hierarchical Models \& the Red Clump"}}
%------------------------------------------------
\entry
{2018 Jul.}
{TASC4/KASC11}
{Aarhus University, Denmark}
{\href{https://speakerdeck.com/ojhall94/testing-asteroseismology-with-gaia-dr2-luminosity-of-the-red-clump}{"Testing asteroseismology with \textit{Gaia} DR2: Luminosity of the Red Clump"}}
%------------------------------------------------
%\entry
%{2017 Apr.}
%{T'DA 2}
%{Aarhus University, Denmark}
%{"Estimating TESS backgrounds with mixture models -- Update"}
%%------------------------------------------------
%\entry
%{2016 Nov.}
%{T'DA 1}
%{University of Birmingham, UK}
%{"Estimating TESS backgrounds with mixture models"}
%%------------------------------------------------
\end{entrylist}

\clearpage
\newgeometry{left=2.5cm,top=1cm,right=2.5cm,bottom=1cm}

%0173b2, de8f05, 029e73, d55e00, cc78bc
\definecolor{c1}{HTML}{d55e00}
\section{\color{c1}conferences \& workshops}

\begin{entrylist}
%%------------------------------------------------
\centrythree
{2021 Mar}
{Cool Stars 20.5}
{\href{http://coolstars20.cfa.harvard.edu/cs20half/program.html}{Virtual}}
%------------------------------------------------
\centrythree
{2021 Feb}
{Streams 21 Workshop}
{\href{https://stellarstreams.org/streams21/}{Virtual}}
%%------------------------------------------------
\centrythree
{2020 Dec}
{SCI Science Workshop 13}
{Virtual - ESA Internal Workshop}
%------------------------------------------------
\centrythree
{2020 Sep.}
{online.TESS.science}
{\href{https://online.tess.science/}{Virtual}}
%------------------------------------------------
\centrythree
{2020 Feb}
{CSH Symposium (\textbf{\textcolor{c1}{invited}})}
{Centre for Space and Habitability, Switzerland}
%------------------------------------------------
\centrythree
{2019 Oct}
{T'DA 9 (\textbf{\textcolor{c1}{invited}})}
{Institute for Astronomy, HI, USA}
%------------------------------------------------
\centrythree
{2019 Aug}
{Astro Hack Week 2019}
{Kavli Institude for Cosmology, UK}
%------------------------------------------------
\centrythree
{2019 Jul}
{TASC5/KASC12  (\textbf{\textcolor{c1}{invited}})}
{MIT, MA, USA}
%------------------------------------------------
\centrythree
{2019 Jan}
{T'DA 8}
{Aarhus University, Denmark}
%------------------------------------------------
\centrythree
{2018 Oct}
{T'DA 5 (\textbf{\textcolor{c1}{invited}})}
{Ohio State University, OH, USA}
%------------------------------------------------
\centrythree
{2018 Jul}
{T'DA 4}
{Aarhus University, Denmark}
%------------------------------------------------
\centrythree
{2018 Jul}
{TASC4/KASC11}
{Aarhus University, Denmark}
%------------------------------------------------
\centrythree
{2018 Jun}
{The Wetton Workshop 2018}
{University of Oxford, UK}
%------------------------------------------------
\centrythree
{2017 Dec}
{T'DA 3}
{KU Leuven, Belgium}
%------------------------------------------------
\centrythree
{2017 Jul}
{TASC3/KASC10}
{University of Birmingham, UK}
%------------------------------------------------
\centrythree
{2017 Apr}
{T'DA 2}
{Aarhus University, Denmark}
%------------------------------------------------
\centrythree
{2016 Nov}
{Asteroseismology of stellar activity cycles}
{Observatoire de la C\^{o}te d'Azur, France}
%------------------------------------------------
\centrythree
{2016 Nov}
{T'DA 1}
{University of Birmingham, UK}
%------------------------------------------------
\end{entrylist}

%0173b2, de8f05, 029e73, d55e00, cc78bc
\definecolor{c1}{HTML}{cc78bc}
\section{\color{c1}posters}

\begin{entrylist}
	%------------------------------------------------
	\centry
	{2021 Mar.}
	{Cool Stars 20.5}
	{\href{http://coolstars20.cfa.harvard.edu/cs20half/program.html}{Virtual}}
	{$+$ \small{\href{https://zenodo.org/record/4562478}{"New asteroseismic rotation rates of Kepler dwarfs show strong agreement with weakened magnetic braking on the late-age main sequence"}}\\
		$+$ \small{\href{https://zenodo.org/record/4562487}{1-minute video 'haiku' shown during the main programme}}}
	%------------------------------------------------
	\centry
	{2020 Dec.}
	{SCI Science Workshop 13}
	{Virtual - ESA Internal Workshop}
	{$+$ \small{"Characterising the Red Clump standard cnalde in magnitude, colour, metallicity and alpha abundance"}\\
		$+$ \small{"New asteroseismic rotation rates of Kepler dwarfs show strong agreement with weakened magnetic braking on the late-age main sequence"}\\
		$+$ \small{1-minute videos accompanying both posters}}
	%------------------------------------------------
	\centry
	{2017 Jul.}
	{TASC5/KASC12}
	{MIT, MA, USA}
	{$+$ \small{"Improving gyrochronology of field stars with asteroseismic age and rotation"}}
	%------------------------------------------------
	\centry
	{2017 Jul.}
	{TASC3/KASC10}
	{University of Birmingham, UK}
	{$+$ \small{"Mixture Models applied to \emph{Kepler} backgrounds \& development for TESS"}}
	%------------------------------------------------
	
\end{entrylist}

%0173b2, de8f05, 029e73, d55e00, cc78bc
\definecolor{c1}{HTML}{0173b2}
\section{\color{c1}research visits}
\begin{entrylist}
	%------------------------------------------------
	\entry
	{2018 Oct.}
	{Visit to the KeplerGO office [3 weeks]}
	{NASA Ames Research Centre, CA, USA}
	{Invited to build the \textbf{\textcolor{c1}{periodogram}} \&  \textbf{\textcolor{c1}{seismology}} modules of \textbf{\textcolor{c1}{\href{http://docs.lightkurve.org/}{Lightkurve}}}.}
	%------------------------------------------------
	\entry
	{2018 Jan.}
	{Visit to SAC [1 week]}
	{Aarhus University, Denmark}
	{Invited to investigate \& build tools for background subtraction of TESS FFIs.}
	%----------------------------------------------
\end{entrylist}

%0173b2, de8f05, 029e73, d55e00, cc78bc
\definecolor{c1}{HTML}{de8f05}
\section{\color{c1}grants \& honours}

\begin{entrylist}
	%------------------------------------------------
	\centrythree
	{2020 \to 2022}
	{ESA Research Fellowship}
	{ESA ESTEC, NL}
	%------------------------------------------------
	\centrythree
	{2019}
	{\textbf{\textcolor{c1}{\pounds 815}} - Ogden Trust Alumni Fund One-Off Grants}
	{The Ogden Trust, UK}
	%------------------------------------------------
	\centrythree
	{2018}
	{\textbf{\textcolor{c1}{\pounds 300}} - IOP Research Student Conference Fund \emph{(declined)}}
	{Institute of Physics, UK}
	
	%------------------------------------------------
	\centrythree
	{2016}
	{\textbf{\textcolor{c1}{\pounds 3000}} - Royal Society Partnership Grant}
	{The Royal Society, UK}
	%------------------------------------------------
	\centrythree
	{2015}
	{Teach Physics Oustanding Intern 2015 - shortlisted}
	{The Ogden Trust, UK}
	%------------------------------------------------
	
\end{entrylist}





\clearpage



%0173b2, de8f05, 029e73, d55e00, cc78bc
\definecolor{c1}{HTML}{029e73}
\section{\color{c1}teaching \& other research}
\begin{entrylist}
	%------------------------------------------------
	\centry
	{2021}
	{LEAPS 2021 \textcolor{c1}{Supervisor}}
	{Virtual - The Leiden/ESA Astrophysics Program for Summer Students}
	{$+$ Primary supervisor for student during a 10-week summer program\\
	 $+$ Jointly ran the selection process, including interviewing a shortlist}
 	%------------------------------------------------
	 \centry
	 {2021 \to now}
	 {Student Supervision}
	 {Virtual - Leiden University}
	 {$+$ Helped advise masters students at the University of Leiden in an unofficial capacity.}
	%------------------------------------------------
	\centry
	{2019}
	{\href{https://www.heacademy.ac.uk/individuals/fellowship}{Advanced HE} - \textcolor{c1}{Associate Fellow} (AFHEA)}
	{Advanced HE}
	{$+$ Formal acknowledgement of teaching experience and expertise}
	%------------------------------------------------
	\centry
	{2019}
	{Access to Birmingham (A2B) supervisor}
	{University of Birmingham}
	{$+$ Supported applicants from disenfranchised backgrounds through the A2B scheme}
	%------------------------------------------------
	\centry
	{2017 \to 2019}
	{2\textsuperscript{nd} Year Laboratory Projects Demonstrator}
	{University of Birmingham, UK}
	{$+$ Taught students to build apparatus and understand their results \\
	$+$ Marked students' work and provided constructive feedback}
	%------------------------------------------------
	\centry
	{2016 \to 2019}
	{3\textsuperscript{rd} Year Observatory Laboratory Supervisor}
	{University of Birmingham, UK}
	{$+$ Supervised students using an observatory and during data reduction\\
	$+$ Helped students understand their results as well as the use of IRAF, Unix, and Python}
	%------------------------------------------------
	\centry
	{2015}
	{Summer Undergraduate Research Experience (SURE)}
	{University of Leicester, UK}
	{$+$ Performed a six-week project using Python to program a robotic arm system for testing a prototype focal plane for the Cherenkov Telescope Array}
	%------------------------------------------------
	\centry
	{2015}
	{Ogden Trust Teach Physics Intern}
	{Bishop Challoner Catholic College, Birmingham, UK}
	{$+$ Helped teach pupils throughout lessons, acting as a teaching assistant\\
	 $+$ Prepared and taught a lesson \& careers workshop}
	%------------------------------------------------
\end{entrylist}

%0173b2, de8f05, 029e73, d55e00, cc78bc
\definecolor{c1}{HTML}{d55e00}
\section{\color{c1}outreach \& engagement}
\begin{entrylist}
	%------------------------------------------------
	\centry
	{2021}
	{\emph{Scientist}, \href{https://www.skypeascientist.com/}{Skype a Scientist}}
	{Virtual}
	{$+$ 2021 Apr - 1st Grade Class, East Lansdowne Elementary, USA\\
		$+$ 2021 Jan - USA-based family, 5th, 3rd and Kindergarten grade
	 }
	%------------------------------------------------
	\centry
	{2021 Apr}
	{\emph{Selected Press} for \textcolor{c1}{Hall et al. 2021}}
	{}
	{$+$ The Independent - \href{https://www.independent.co.uk/life-style/gadgets-and-tech/star-spinning-astronomy-seismology-b1835897.html}{"Old stars are not behaving as expected, scientists say"}\\
	$+$ Metro - \href{https://metro.co.uk/2021/04/22/stars-spin-faster-as-they-get-older-astronomers-learn-14453841/}{"Stars spin faster as they get older, astronomers learn"}
	} 
	%------------------------------------------------
	\centry
	{2021 Mar}
	{\emph{Speaker}, Astronomy on Tap Leiden}
	{Leiden, NL}
	{\href{https://youtu.be/lreGHVBKfjo?t=1925}{A recording of the talk is \textbf{\textcolor{c1}{available online.}}}}	
	%------------------------------------------------
	\centry
	{2019 \to 2021}
	{\emph{Author}, \href{https://astrobites.org/}{Astrobites Collaboration}}
	{}
	{$+$ Wrote and edited monthly summaries of astronomy papers at an undergraduate level. \\ Committee member for \textbf{\textcolor{c1}{Advertising}}, \textbf{\textcolor{c1}{Moderating}}, \textbf{\textcolor{c1}{Hiring}}, \textbf{\textcolor{c1}{Undergraduate Engagement}}, and \textbf{\textcolor{c1}{Equality, Diversity \& Inclusion}}}
	%------------------------------------------------
	\centry
	{2019}
	{\emph{Developer}, \href{https://github.com/ojhall94/stateoftheuniverse}{State of The Universe} collaboration}
	{Astro Hack Week 2019}
	{$+$ Helped build and maintain an informative package for teachers and planetarium guides.}
	%------------------------------------------------
	\centry
	{2018 \to 2019}
	{\emph{Demonstrator}, Applicant Visit Days}
	{University of Birmingham, UK}
	{$+$ Developed and taught laboratory sessions for undergraduate applicants.}
	%------------------------------------------------
	\centry
	{2016 \to 2017}
	{\emph{Partnered Researcher}, \textcolor{c1}{Royal Society Partnership Grant}}
	{Bishop Challoner Catholic College, UK}
	{$+$ Developed and taught a series of lessons and lab activities engaging Year 9 pupils with exoplanet characterisation and asteroseismology.}
\end{entrylist}

\clearpage

%0173b2, de8f05, 029e73, d55e00, cc78bc
\definecolor{c1}{HTML}{cc78bc}
\section{\color{c1}community services}
\begin{entrylist}
	%------------------------------------------------	
	\centry
	{2021}
	{\emph{\textcolor{c1}{Panelist}}, TESS Cycle 4}
	{Virtual - NASA Goddard}
	{$+$ Collaborated virtually with a global team of panelists to rank research proposals}
	%------------------------------------------------	
		\centry
	{2020 \to now}
	{\emph{\textcolor{c1}{Reviewer}}}
	{}
	{$+$ For The Astrophysical Journal}
	%------------------------------------------------		
	\centry
	{2020}
	{\emph{LOC}, SCI Science Workshop 13}
	{Virtual - ESA Internal Workshop}
	{$+$ Organised poster viewing and social gatherings in \href{https://gather.town/}{Gather Town}\\
		$+$ Moderated speaker sessions}


	%------------------------------------------------
	\centry
	{2018 \to 2019}
	{\emph{Organiser}, 9\textsuperscript{th} BEAR Conference}
	{University of Birmingham, UK}
	{$+$ Organised local annual high performance computing conference.}	
	%------------------------------------------------	
	\centry
	{2017}
	{\emph{LOC}, TASC3/KASC11}
	{University of Birmingham, UK}
	{$+$ Helped organise 150+ attendee asteroseismology conference.}
	%------------------------------------------------

	\centrythree
	{2018 \to now}
	{Member of the \textbf{\textcolor{c1}{Lightkurve}} collaboration}
	{NASA Ames Research Centre, CA, USA}
	%------------------------------------------------	
	\centrythree
	{2016 \to now}
	{Member of the \emph{TESS Data for Asteroseismology} (\textbf{\textcolor{c1}{T'DA}}) collaboration }
	{}
	%------------------------------------------------
	\centrythree
	{2016 \to now}
	{Member of the \emph{TESS Asteroseismic Science Consortium} (TASC)}
	{}
	%------------------------------------------------
	\centrythree
	{2016 \to now}
	{Member of the \emph{International Astronomical Union} (IAU)}
	{}	
	%------------------------------------------------	
\end{entrylist}



\vspace{0.5cm}
%0173b2, de8f05, 029e73, d55e00, cc78bc
\definecolor{c1}{HTML}{0173b2}
\section{\color{c1}selected publications}

\bodyfont 20 publications, of which 2 as first author, with 302 total citations. \textbf{\textcolor{c1}{H-index:} 9}

\textbf{\color{c1}{first \& second author publications:}}
\vspace{-0.2cm}
\begin{enumerate}
	\item \bibentry{\textbf{\color{c1}Hall, O. J.}, Davies, G. R., van Saders, J. and 9 coauthors}
	{Weakened magnetic braking supported by asteroseismic rotation rates of Kepler dwarfs}
	{\textbf{\textcolor{c1}{Nature Astronomy}}, \textbf{2021}}
	{\textit{Summary}: Made new measurements of asteroseismic rotation rates, and compared these to population models of rotational evolution to indicate the presence of weakened magnetic braking.}
	{\texttt{\href{https://doi.org/10.1038/s41550-021-01335-x}{doi:10.1038/s41550-021-01335-x}, \href{https://arxiv.org/abs/2104.10919}{arXiv:2104.10919}}}
	
	\item \bibentry{\textbf{\color{c1}Hall, O. J.}, Davies, G. R., Elsworth, Y. P. and 9 coauthors}
	{Testing asteroseismology with \textit{Gaia} DR2: Hierarchical models of the Red Clump}
	{Monthly Notices of the Royal Astronomical Society, \textbf{2019}}
	{\textit{Summary}: Constrained the luminosity of the Red Clump and the \textit{Gaia} DR2 parallax zero-point offset simultaneously using hierarchical latent variable models.}
	{\texttt{\href{https://academic.oup.com/mnras/article-abstract/486/3/3569/5475128}{doi:10.1093/mnras/stz1092}, \href{https://arxiv.org/abs/1904.07919}{arXiv:1904.07919}}}

	\item \bibentry{Khan, S., \textbf{\color{c1}Hall, O. J.}, Miglio, A., Davies, G. R., Mosser, B., Girardi, L., Montalb\'an, J.}
{The Red-giant Branch Bump Revisited: Constraints on Envelope Overshooting in a Wide Range of Masses and Metallicities}
{The Astrophysical Journal, \textbf{2018}}
{\textit{Contribution:} Used Mixture Models to constrain the position of the Red-Giant Branch Bump.}
{\texttt{\href{https://iopscience.iop.org/article/10.3847/1538-4357/aabf90}{doi:10.3847/1538-4357/aabf90}, \href{https://arxiv.org/abs/1804.06669}{arXiv:1804.06669}}}
\end{enumerate}


\textbf{\color{c1}{contributing author publications:}}
\vspace{-0.2cm}
\begin{enumerate}
	\setcounter{enumi}{3}
	\item \bibentry{Montalb\'{a}n, J., Mackereth, J. T., Miglio, A. and 16 coauthors including \textbf{\color{c1}Hall, O. J.}}
	{Chronologically dating the early assembly of the Milky Way}
	{Accepted, \textbf{\textcolor{c1}{Nature Astronomy}}, \textbf{2021}}
	{\bodyfont \textit{Contribution:} Obtained seismic parameters for stellar sample and helped develop hierachical model.}
	{\texttt{\href{https://arxiv.org/abs/2001.04653}{arxiv:2001.04653}}}
	
	\item \bibentry{Mackereth, J. T., Miglio, A., Elsworth, Y., and 30 coauthors including \textbf{\color{c1}Hall, O. J.}}
	{Prospects for Galactic and stellar astrophysics with asteroseismology of giant stars in the TESS continuous viewing zones and beyond}
	{Monthly Notices of the Royal Astronomical Society, \textbf{2021}}
	{\bodyfont \textit{Contribution:} Obtained fundamental seismic parameters for stellar sample.}
	{\texttt{\href{https://doi.org/10.1093/mnras/stab098}{doi:10.1093/mnras/stab098}, \href{https://arxiv.org/abs/2012.00140}{arXiv:2012.00140}}}	

	\item \bibentry{Nielsen, M. B., Davies, G. R., Ball, W. H., Lyttle, A. J., Li, T., \textbf{\textcolor{c1}{Hall, O. J.}} and 11 other coauthors}
	{PBjam: A Python Package for Automating Asteroseismology of Solar-like Oscillators}
	{The Astronomical Journal, \textbf{2021}}
	{\bodyfont \textit{Contribution:} Developed code and documentation for PBJam package}
	{\texttt{\href{https://iopscience.iop.org/article/10.3847/1538-3881/abcd39}{doi:10.3847/1538-3881/abcd39}, \href{https://arxiv.org/abs/2012.00580}{arXiv:2012.00580}}}	

	\item \bibentry{Silva Aguirre, V., Stello, D., Stokholm, A. and 75 coauthors including \textbf{\color{c1}Hall, O. J.}}
	{Detection and characterisation of oscillating red giants: first results from the TESS satellite}
	{The Astrophysical Journal, \textbf{2020}}
	{\bodyfont \textit{Contribution:} Obtained fundamental seismic parameters for stellar sample.}
	{\texttt{\href{https://iopscience.iop.org/article/10.3847/2041-8213/ab6443}{doi:10.3847/2041-8213/ab6443}, \href{https://arxiv.org/abs/1912.07604}{arXiv:1912.07604}}}

	\item \bibentry{Chaplin, W., Serenelli, A. M., Miglio, A. and 83 coauthors including \textbf{\color{c1}Hall, O. J.}}
	{Age dating of an early Milky Way merger via asteroseismology of the naked-eye star $\nu$Indi}
	{\textbf{\textcolor{c1}{Nature Astronomy}}, \textbf{2020}}
	{\bodyfont \textit{Contribution:} Advised on systematic uncertainties in spectroscopic methods.}
	{\texttt{\href{https://www.nature.com/articles/s41550-019-0975-9}{doi:10.1038/s41550-019-0975-9},  \href{https://arxiv.org/abs/2001.04653}{arXiv:2001.04653}}}
	
	\item \bibentry{{Huber}, D., {Chaplin}, W. J., {Chontos}, A and 139 coauthors including \textbf{\color{c1}Hall, O. J.}}
	{{A Hot Saturn Orbiting An Oscillating Late Subgiant Discovered by TESS}}
	{The Astronomical Journal, \textbf{2019}}
	{\textit{Contribution}: Checked proper use and interpretation of \textit{Gaia} parallaxes.}	
	{\texttt{\href{https://iopscience.iop.org/article/10.3847/1538-3881/ab1488}{doi:10.3847/1538-3881/ab1488}, \href{https://arxiv.org/abs/1901.01643}{arXiv:1901.01643}}}

	\item \bibentry{{Bugnet}, L., {Garc\'{i}a}, R. A., {Mathur}, S.,
		{Davies}, G. R., \textbf{\color{c1}{Hall}, O. J.}, {Lund}, M. N., {Rendle}, B. M.}
	{FliPer$_{Class}$: In search of solar-like pulsators among TESS targets}
	{Astronomy \& Astrophysics, \textbf{2019}}
	{\textit{Contribution}: Aided with interpretation of systematic uncertainties on effective temperature.}	
	{\texttt{\href{https://www.aanda.org/articles/aa/abs/2019/04/aa34780-18/aa34780-18.html}{doi:10.1051/0004-6361/201834780}, \href{https://arxiv.org/abs/1902.09854}{arXiv:1902.09854}}}
	
	\item \bibentry{Bugnet, L., Garc\'{i}a, R. A., Davies, G. R., Mathur, S., Corsaro, E., \textbf{\color{c1}Hall, O. J.}, Rendle, B. M.}
	{FliPer: A global measure of power density to estimate surface gravities of main-sequence solar-like stars and red giants}
	{Astronomy \& Astrophysics, \textbf{2018}}
	{\textit{Contribution:} Helped develop the FliPer metric \& its machine learning implementation.}
	{\texttt{\href{https://www.aanda.org/articles/aa/abs/2018/12/aa33106-18/aa33106-18.html}{doi:0.1051/0004-6361/201833106}, \href{https://arxiv.org/abs/1809.05105}{arXiv:1809.05105}}}	
		
	\item \bibentry{{Davies}, G. R., {Lund}, M. N., {Miglio}, A., Elsworth, Y. P. and 13 coauthors including \textbf{\color{c1}Hall, O. J.}}
	{Using red clump stars to correct the \emph{Gaia} DR1 parallaxes}
	{Astronomy \& Astrophysics, \textbf{2017}}
	{\textit{Contribution}: Verified results found by lead authors.}	
	{\texttt{\href{https://www.aanda.org/articles/aa/abs/2017/02/aa30066-16/aa30066-16.html}{doi:10.1051/0004-6361/201630066}, \href{https://arxiv.org/abs/1701.02506}{arXiv:1701.02506}}}
\end{enumerate}


\textbf{\color{c1}{software publications:}}
\vspace{-0.2cm}
\begin{enumerate}
	\setcounter{enumi}{12}
	\item \bibentry{{Lightkurve Collaboration}, {Cardoso}, J. V. d. M., {Hedges}, C., Gully-Santiago, M., Saunders, N., Cody, A-M., Barclay, T., \textbf{\color{c1}Hall, O. J.}, Sagear, S., Turtelboom, E., Zhang, J., Tzanidakis, A., Mighell, K., Coughlin, J., Bell, K., Berta-Thompson, Z., Williams, P., Dotson, J., Barentsen, G.}
	{Lightkurve: Kepler and TESS time series analysis in Python}
	{Astrophysics Source Code Library, \textbf{2018}}
	{\textit{Contribution:} Led development of the `periodogram' and `seismology' modules.}
	{\texttt{\href{http://ascl.net/1812.013}{ascl:1812.013}}}
\end{enumerate}


\textbf{\color{c1}{white papers:}}
\vspace{-0.2cm}
\begin{enumerate}
		\setcounter{enumi}{13}
	\item \bibuf{{Khullar}, G., {Kholer}, S., {Konchady}, T.  and 32 coauthors including \textbf{\color{c1}Hall, O. J.}}
	{Astrobites as a Community-led Model for Education, Science Communication, and Accessibility in Astrophysics}
	{arXiv e-prints, \textbf{2019}}
	{\texttt{\href{https://arxiv.org/abs/1907.09496}{arXiv:1907.09496}}}
\end{enumerate}

\end{document}
