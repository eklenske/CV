%%%%%%%%%%%%%%%%%%%%%%%%%%%%%%%%%%%%%%%%%
% 
% K-CV -- Klenske Curriculum Vitae
% 
% Simplified Friggeri CV 
% by Edgar Klenske (ed.klenske@gmx.de)
% https://github.com/eklenske/CV
%
% Forked from:
% Jelmer Tiente 
% https://github.com/JelmerT/CV
%
% Based on original work by:
% Adrien Friggeri (adrien@friggeri.net)
% https://github.com/afriggeri/CV
%
% License:
% CC BY-NC-SA 3.0 (http://creativecommons.org/licenses/by-nc-sa/3.0/)
%
%%%%%%%%%%%%%%%%%%%%%%%%%%%%%%%%%%%%%%%%%

\documentclass[letterpaper]{k-cv} % Add 'print' as an option into the square
                       % bracket to remove colors from this template
                       % for printing
\usepackage[T1]{fontenc}
\newcommand\tab[1][1cm]{\hspace*{#1}}

\definecolor{c1}{HTML}{0173b2}
\definecolor{c2}{HTML}{de8f05}
\definecolor{c3}{HTML}{029e73}
\definecolor{c4}{HTML}{d55e00}
\pagenumbering{arabic}

\begin{document}
\header{Oliver James }{Hall}{ESA Research Fellow - Asteroseismology \& Statistics} % Your name and current job

%-------------------------------------------------------------------------------
%	SIDEBAR SECTION
%-------------------------------------------------------------------------------

\begin{aside} % In the aside, each new line forces a line break
\section{\color{c1}programming}
\bodyfont Python, Git
Unix, LaTeX,
SQL
\section{\color{c2}skills}
Stan
PyMC3
emcee 
Bayesian statistics
Hierarchical models
Asteroseismology
Jupyter Notebooks
Science Communication \& Writing
Software development \& publication
\section{\color{c3}languages}
English, Dutch (bilingual)
\section{\color{c4}contact}
European Space Research \& Technology Centre
Keplerlaan 1
Postbus 299
2200 AG Noordwijk
The Netherlands
~
\href{mailto:ojhall94@gmail.com}{ojhall94@gmail.com}
\href{http://www.ojhall94.github.io}{ojhall94.github.io}
\href{http://www.github.com/ojhall94}{GitHub/ojhall94}
\href{http://www.twitter.com/asteronomer}{@asteronomer}
\href{http://www.orcid.com/0000-0002-0468-4775}{ORCID/}
\href{http://www.orcid.com/0000-0002-0468-4775}{0000-0002-0468-4775}
Tel: (+31)(0)614227748
\end{aside}

\section{\color{c1}presentations}

\begin{entrylist}
%------------------------------------------------
\entry
{2019 Nov.}
{\textbf{\textcolor{c1}{Seminar}}}
{University of Exeter, UK}
{"Asteroseismology \& Applied Statistics"}
%------------------------------------------------
\entry
{2019 Jul.}
{TASC5/KASC12}
{MIT, MA, USA}
{\textbf{\textcolor{c1}{Invited talk:}} "Accessible Asteroseismology with Lightkurve"\\ \small{Poster: "Improving gyrochronology of field stars with asteroseismic age and rotation"}}
%------------------------------------------------
\entry
{2018 Dec.}
{Birmingham-Warwick Science Meet-Up}
{University of Warwick, UK}
{"Testing asteroseismology with \textit{Gaia} DR2: Hierarchical Models \& the Red Clump"}
%------------------------------------------------
\entry
{2018 Jul.}
{TASC4/KASC11}
{Aarhus University, Denmark}
{"Testing asteroseismology with \textit{Gaia} DR2: Luminosity of the Red Clump"}
%------------------------------------------------
\entry
{2017 Jul.}
{TASC3/KASC10}
{University of Birmingham, UK}
{\small{Poster: "Mixture Models applied to \emph{Kepler} backgrounds \& development for TESS"}}
%------------------------------------------------
\entry
{2017 Apr.}
{T'DA 2}
{Aarhus University, Denmark}
{"Estimating TESS backgrounds with mixture models -- Update"}
%------------------------------------------------
\entry
{2016 Nov.}
{T'DA 1}
{University of Birmingham, UK}
{"Estimating TESS backgrounds with mixture models"}
%------------------------------------------------
\end{entrylist}

\section{\color{c2}conferences \& workshops}

\begin{entrylist}
%------------------------------------------------
\entrythree
{2019 Oct.}
{T'DA 9 (\textbf{\textcolor{c2}{invited}})}
{Institute for Astronomy, HI, USA}
%------------------------------------------------
\entrythree
{2019 Aug.}
{Astro Hack Week 2019}
{Kavli Institude for Cosmology, UK}
%------------------------------------------------
\entrythree
{2019 Jul.}
{TASC5/KASC12  (\textbf{\textcolor{c2}{invited}})}
{MIT, MA, USA}
%------------------------------------------------
\entrythree
{2019 Jan.}
{T'DA 8}
{Aarhus University, Denmark}
%------------------------------------------------
\entrythree
{2018 Oct.}
{T'DA 5 (\textbf{\textcolor{c2}{invited}})}
{Ohio State University, OH, USA}
%------------------------------------------------
\entrythree
{2018 Jul.}
{T'DA 4}
{Aarhus University, Denmark}
%------------------------------------------------
\entrythree
{2018 Jul.}
{TASC4/KASC11}
{Aarhus University, Denmark}
%------------------------------------------------
\entrythree
{2018 Jun.}
{The Wetton Workshop 2018}
{University of Oxford, UK}
%------------------------------------------------
\entrythree
{2017 Dec.}
{T'DA 3}
{KU Leuven, Belgium}
%------------------------------------------------
\entrythree
{2017 Jul.}
{TASC3/KASC10}
{University of Birmingham, UK}
%------------------------------------------------
\entrythree
{2017 Apr.}
{T'DA 2}
{Aarhus University, Denmark}
%------------------------------------------------
\entrythree
{2016 Nov.}
{Asteroseismology of stellar activity cycles}
{Observatoire de la C\^{o}te d'Azur, France}
%------------------------------------------------
\entrythree
{2016 Nov.}
{T'DA 1}
{University of Birmingham, UK}
%------------------------------------------------
\end{entrylist}


\section{\color{c3}research visits}
\begin{entrylist}
%------------------------------------------------
\entry
{2018 Oct.}
{Visit to the KeplerGO office [3 weeks]}
{NASA Ames Research Centre, CA, USA}
{Invited to build the \textbf{\textcolor{c3}{periodogram}} \&  \textbf{\textcolor{c3}{seismology}} modules of \textbf{\textcolor{c3}{\href{http://docs.lightkurve.org/}{Lightkurve}}}.}
%------------------------------------------------
\entry
{2018 Jan.}
{Visit to SAC [1 week]}
{Aarhus University, Denmark}
{Invited to investigate \& build tools for background subtraction of TESS FFIs.}
%----------------------------------------------
\end{entrylist}

\section{\color{c4}grants \& awards}

\begin{entrylist}
	%------------------------------------------------
	\entrythree
	{2019}
	{\textbf{\textcolor{c4}{\pounds 815}} - Ogden Trust Alumni Fund One-Off Grants}
	{The Ogden Trust, UK}
	%------------------------------------------------
	\entrythree
	{2018}
	{\textbf{\textcolor{c4}{\pounds 300}} - IOP Research Student Conference Fund \emph{(declined)}}
	{Institute of Physics, UK}
	
	%------------------------------------------------
	\entrythree
	{2016}
	{\textbf{\textcolor{c4}{\pounds 3000}} - Royal Society Partnership Grant}
	{The Royal Society, UK}
	%------------------------------------------------
	\entrythree
	{2015}
	{Teach Physics Oustanding Intern 2015 - shortlisted}
	{The Ogden Trust, UK}
	%------------------------------------------------
	
\end{entrylist}


%\clearpage
%\smallheader{oliver james  }{hall}
\newgeometry{left=2.5cm,top=1cm,right=2.5cm,bottom=1cm}
%\clearpage
%\newgeometry{left=2.5cm,top=1cm,right=2.5cm,bottom=1cm}

\section{\color{c1}education \& employment}

\begin{entrylist}
\centry
{2020 \to now}
{ESA Research Fellow}
{European Space Research \& Technology Centre, Netherlands}
{Supervisor: Ana Heras}
	
\centry
{2016 \to 2020}
{PhD {\normalfont in Physics \& Astronomy}}
{University of Birmingham, UK}
{Supervisor: Dr. Guy R. Davies\\
\textit{Thesis}: "Ensemble Asteroseismology and Hierarchical Bayesian Models: New Inferences of Astrophysics with Oscillating Stars"} 
%------------------------------------------------
%------------------------------------------------
\centry
{2012 \to 2016}
{M.Sci. {\normalfont Physics \& Astrophysics}}
{University of Birmingham, UK}
{Dissertation supervisor: Prof. William J. Chaplin\\
1\textsuperscript{st} Class w. Honours\\ \textit{Thesis}: "Detecting Signatures of Stellar Activity Cycles in Solar-Type Stars Using Asteroseismic Analysis of P-Mode Amplitude Shifts"}
%------------------------------------------------



%------------------------------------------------
\centry
{2006 \to 2012}
{Gymnasium}
{Gemeentelijk Gymnasium Hilversum, Netherlands}
{8.5/10 average across eleven subjects}
%------------------------------------------------
\end{entrylist}


\section{\color{c2}teaching and research}
\begin{entrylist}
	%------------------------------------------------
	\centry
	{2019}
	{\href{https://www.heacademy.ac.uk/individuals/fellowship}{Advanced HE} - \textcolor{c2}{Associate Fellow} (AFHEA)}
	{Advanced HE}
	{}
	%------------------------------------------------
	\centry
	{2019}
	{Access to Birmingham (A2B) supervisor}
	{University of Birmingham}
	{Supported applicants from disenfranchised backgrounds through the A2B scheme.}
	%------------------------------------------------
	\centry
	{2017 \to 2019}
	{2\textsuperscript{nd} Year Laboratory Projects Demonstrator}
	{University of Birmingham, UK}
	{Taught students to build apparatus and understand their results. I marked their work and provided constructive feedback.}
	%------------------------------------------------
	\centry
	{2016 \to 2019}
	{3\textsuperscript{rd} Year Observatory Laboratory Supervisor}
	{University of Birmingham, UK}
	{Supervised students using an observatory. Helped students understand their results as well as the use of IRAF, Unix, and Python.}
	%------------------------------------------------
	\centry
	{2015}
	{Summer Undergraduate Research Experience (SURE)}
	{University of Leicester, UK}
	{Performed a six-week project using Python to program a robotic arm system for testing a prototype focal plane for the Cherenkov Telescope Array.}
	%------------------------------------------------
	\centry
	{2015}
	{Ogden Trust Teach Physics Intern}
	{Bishop Challoner Catholic College, Birmingham, UK}
	{Helped teach pupils throughout lessons, prepared and taught a lesson \& careers workshop of my own design.}
	%------------------------------------------------
\end{entrylist}

\section{\color{c3}outreach \& engagement}
\begin{entrylist}
	%------------------------------------------------
	\centry
	{2019 \to now}
	{\emph{Author}, Astrobites Collaboration}
	{}
	{Write and edit monthly summaries of astronomy papers at an undergraduate level. \\ Committee member for \textbf{\textcolor{c3}{Advertising}}, \textbf{\textcolor{c3}{Moderating}}, \textbf{\textcolor{c3}{Hiring}}, \textbf{\textcolor{c3}{Undergraduate Engagement}}, and \textbf{\textcolor{c3}{Equality, Diversity \& Inclusion}}}
	%------------------------------------------------
	\centry
	{2019}
	{\emph{Developer}, State of The Universe collaboration}
	{Astro Hack Week 2019}
	{Helped build and maintain an informative package for teachers and planetarium guides.}
	%------------------------------------------------
	\centry
	{2018 \to 2019}
	{\emph{Organiser}, 9\textsuperscript{th} BEAR Conference}
	{University of Birmingham, UK}
	{Organised local annual high performance computing conference.}
	%------------------------------------------------
	\centry
	{2018 \to 2019}
	{\emph{Demonstrator}, Applicant Visit Day}
	{University of Birmingham, UK}
	{Developed and taught laboratory sessions for undergraduate applicants.}
	%------------------------------------------------
	\centry
	{2016 \to 2017}
	{\emph{Partnered Researcher}, \textcolor{c3}{Royal Society Partnership Grant}}
	{}
	{Developed and taught a series of lessons and lab activities engaging Year 9 pupils with exoplanet characterisation and asteroseismology.}
\end{entrylist}

\clearpage
\newgeometry{left=2.5cm,top=1cm,right=2.5cm,bottom=1cm}

\section{\color{c4}community services}
\begin{entrylist}
	%------------------------------------------------
	\centrythree
	{2018 \to now}
	{Member of the \textbf{\textcolor{c4}{Lightkurve}} collaboration}
	{NASA Ames Research Centre, CA, USA}
	%------------------------------------------------
	\centrythree
	{2016 \to now}
	{Member of the \emph{TESS Data for Asteroseismology} (\textbf{\textcolor{c4}{T'DA}}) collaboration }
	{}
	%------------------------------------------------
	\centrythree
	{2016 \to now}
	{Member of the \emph{TESS Asteroseismic Science Consortium} (TASC)}
	{}
	%------------------------------------------------
	\centrythree
	{2017}
	{LOC member for TASC3/KASC11}
	{University of Birmingham, UK}
	%------------------------------------------------
\end{entrylist}

\definecolor{c2}{HTML}{0173b2}

\vspace{0.5cm}
\section{\color{c2}selected publications}
\vspace{-0.5cm}
\subsubsection*{\color{c2}first author publications:}
\vspace{-0.2cm}
\begin{enumerate}
	\item \bibentry{\textbf{\color{c2}Hall, O. J.}, Davies, G. R., Elsworth, Y. P. and 9 coauthors}
	{Testing asteroseismology with \textit{Gaia} DR2: Hierarchical models of the Red Clump}
	{Monthly Notices of the Royal Astronomical Society, 2019}
	{\textit{Summary}: Constrained the luminosity of the Red Clump and the \textit{Gaia} DR2 parallax zero-point offset simultaneously using hierarchical latent variable models.}
	{\texttt{\href{https://academic.oup.com/mnras/article-abstract/486/3/3569/5475128}{doi:10.1093/mnras/stz1092}, \href{https://arxiv.org/abs/1904.07919}{arXiv:1904.07919}}}
\end{enumerate}

\subsubsection*{\color{c2}contributing author publications:}
\begin{enumerate}
	\setcounter{enumi}{1}
	\item \bibentry{Khan, S., \textbf{\color{c2}Hall, O. J.}, Miglio, A., Davies, G. R., Mosser, B., Girardi, L., Montalb\'an, J.}
	{The Red-giant Branch Bump Revisited: Constraints on Envelope Overshooting in a Wide Range of Masses and Metallicities}
	{The Astrophysical Journal, 2018}
	{\textit{Contribution:} Used Mixture Models to constrain the position of the Red-Giant Branch Bump.}
	{\texttt{\href{https://iopscience.iop.org/article/10.3847/1538-4357/aabf90}{doi:10.3847/1538-4357/aabf90}, \href{https://arxiv.org/abs/1804.06669}{arXiv:1804.06669}}}

	\item \bibentry{Bugnet, L., Garc\'{i}a, R. A., Davies, G. R., Mathur, S., Corsaro, E., \textbf{\color{c2}Hall, O. J.}, Rendle, B. M.}
	{FliPer: A global measure of power density to estimate surface gravities of main-sequence solar-like stars and red giants}
	{Astronomy \& Astrophysics, 2018}
	{\textit{Contribution:} Helped develop the FliPer metric \& its machine learning implementation.}
	{\texttt{\href{https://www.aanda.org/articles/aa/abs/2018/12/aa33106-18/aa33106-18.html}{doi:0.1051/0004-6361/201833106}, \href{https://arxiv.org/abs/1809.05105}{arXiv:1809.05105}}}
	
	\item \bibentry{Silva Aguirre, V., Stello, D., Stokholm, A. and 75 coauthors including \textbf{\color{c2}Hall, O. J.}}
	{Detection and characterisation of oscillating red giants: first results from the TESS satellite}
	{The Astrophysical Journal}
	{\bodyfont \textit{Contribution:} Obtained fundamental seismic parameters for stellar sample.}
	{\texttt{\href{https://iopscience.iop.org/article/10.3847/2041-8213/ab6443}{doi:10.3847/2041-8213/ab6443}, \href{https://arxiv.org/abs/1912.07604}{arXiv:1912.07604}}}
		
	\item \bibentry{Montalb\'{a}n, J., Mackereth, J. T., Miglio, A. and 16 coauthors including \textbf{\color{c2}Hall, O. J.}}
	{Chronologically dating the early assembly of the Milky Way}
	{arxiv e-prints, 2020. Under review for publication in Nature Astronomy}
	{\bodyfont \textit{Contribution:} Obtained seismic parameters for stellar sample and helped develop hierachical model.}
	{\texttt{\href{arxiv:2001.04653}{https://arxiv.org/abs/2001.04653}}}
		
	\item \bibentry{Chaplin, W., Serenelli, A. M., Miglio, A. and 83 coauthors including \textbf{\color{c2}Hall, O. J.}}
	{Age dating of an early Milky Way merger via asteroseismology of the naked-eye star $\nu$Indi}
	{Nature Astronomy}
	{\bodyfont \textit{Contribution:} Advised on systematic uncertainties in spectroscopic methods.}
	{\texttt{\href{https://www.nature.com/articles/s41550-019-0975-9}{doi:10.1038/s41550-019-0975-9},  \href{https://arxiv.org/abs/2001.04653}{arXiv:2001.04653}}}
%	
	\item \bibentry{{Bugnet}, L., {Garc\'{i}a}, R. A., {Mathur}, S.,
		{Davies}, G. R., \textbf{\color{c2}{Hall}, O. J.}, {Lund}, M. N., {Rendle}, B. M.}
	{FliPer$_{Class}$: In search of solar-like pulsators among TESS targets}
	{arXiv e-prints, 2019}
	{\textit{Contribution}: Aided with interpretation of systematic uncertainties on effective temperature.}	
	{\texttt{\href{https://www.aanda.org/articles/aa/abs/2019/04/aa34780-18/aa34780-18.html}{doi:10.1051/0004-6361/201834780}, \href{https://arxiv.org/abs/1902.09854}{arXiv:1902.09854}}}
	
	\item \bibentry{{Huber}, D., {Chaplin}, W. J., {Chontos}, A and 139 coauthors including \textbf{\color{c2}Hall, O. J.}}
	{{A Hot Saturn Orbiting An Oscillating Late Subgiant Discovered by TESS}}
	{arXiv e-prints, 2019}
	{\textit{Contribution}: Checked proper use and interpretation of \textit{Gaia} parallaxes.}	
	{\texttt{\href{https://iopscience.iop.org/article/10.3847/1538-3881/ab1488}{doi:10.3847/1538-3881/ab1488}, \href{https://arxiv.org/abs/1901.01643}{arXiv:1901.01643}}}
	
	\item \bibentry{{Davies}, G. R., {Lund}, M. N., {Miglio}, A., Elsworth, Y. P. and 13 coauthors including \textbf{\color{c2}Hall, O. J.}}
	{Using red clump stars to correct the \emph{Gaia} DR1 parallaxes}
	{Astronomy \& Astrophysics, 2017}
	{\textit{Contribution}: Verified results found by lead authors.}	
	{\texttt{\href{https://www.aanda.org/articles/aa/abs/2017/02/aa30066-16/aa30066-16.html}{doi:10.1051/0004-6361/201630066}, \href{https://arxiv.org/abs/1701.02506}{arXiv:1701.02506}}}
\end{enumerate}
%
%\smallheader{Oliver James }{Hall}
\vspace{-0.2cm}
\subsubsection*{\color{c2}software publications:}
\vspace{-0.2cm}
\begin{enumerate}
	\setcounter{enumi}{8}
	\item \bibentry{{Lightkurve Collaboration}, {Cardoso}, J. V. d. M., {Hedges}, C., Gully-Santiago, M., Saunders, N., Cody, A-M., Barclay, T., \textbf{\color{c2}Hall, O. J.}, Sagear, S., Turtelboom, E., Zhang, J., Tzanidakis, A., Mighell, K., Coughlin, J., Bell, K., Berta-Thompson, Z., Williams, P., Dotson, J., Barentsen, G.}
	{Lightkurve: Kepler and TESS time series analysis in Python}
	{Astrophysics Source Code Library, 2018}
	{\textit{Contribution:} Led development of the `periodogram' and `seismology' modules.}
	{\texttt{\href{http://ascl.net/1812.013}{ascl:1812.013}}}
\end{enumerate}

\vspace{-0.2cm}
\subsubsection*{\color{c2}white papers:}
\vspace{-0.2cm}
\begin{enumerate}
		\setcounter{enumi}{9}
	\item \bibuf{{Khullar}, G., {Kholer}, S., {Konchady}, T.  and 32 coauthors including \textbf{\color{c2}Hall, O. J.}}
	{Astrobites as a Community-led Model for Education, Science Communication, and Accessibility in Astrophysics}
	{arXiv e-prints, 2019}
	{\texttt{\href{https://arxiv.org/abs/1907.09496}{arXiv:1907.09496}}}
\end{enumerate}

\end{document}
